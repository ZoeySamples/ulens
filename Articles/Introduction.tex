\documentclass{article}

\usepackage{graphicx}

\begin{document}

Microlensing is becoming an increasingly popular method to search for
planets because of its ability to discover more diverse distributions
of planets than other common methods. Whereas the radial velocity method
and the transit method are most sensitive to hot Jupiters, microlensing
is most sensitive to distant Neptunes. Moreover, it has the potential
to detect Earth-sized and even Mercury-sized planets, something the other
methods are not yet able to do. The Wide Field Infrared Survey Telescope
(WFIRST) is expected to be launched in the mid-2020s. Astronomers are
planning on focusing the WFIRST on the galactic bulge, where it will
measure light curves for microlensing events. By doing this, researchers
hope to discover more properties of planet formation, including for
Earth-like planets.

A microlensing event occurs when a distant body, known as the source,
passes behind another body or system of bodies, known as the lens. By
the gravitational lens effect, the light of the source is bent around
the lens and magnified. This produces an observable light curve that can
indicate details about the lensing bodies, including the mass ratios and
separations between each of the lensing bodies. My project focus on 
modeling point-source point-lens microlensing events for binary and triple
lenses.

In a binary lens, the mass ratio of the planet to the star is denoted by $q$,
and the separation between the planet and the star is denoted by $s$. The
separation is in units of the Einstein radius of the whole lens system, 
which means $q$ and $s$ are both unitless. In a triple lens, there will be two
mass ratios, two separations, and an angle that is formed by the 3 bodies, 
which all have various definitions depending on the types of bodies in the
system (e.g. star(s), planet(s), or a moon) and the parameter space being used.
The definitions of these parameters are arbitrarily defined on a case-by-case
basis, so they will not be defined here.

The lensing equation is given by
\begin{equation}
\zeta = z + \sum_{i} \frac{m_{i}}{\bar{z_{i}} - \bar{z}}
\end{equation}
where ${m_{i}}$ is the mass of the lensing body divided by the total mass
of the system, ${z_{i}}$ is the position of the lensing body in units of the
total system's Einstein radius, $z$ is the position of each of the resulting 
images, $\bar{z}$ is the complex conjugate of $z$, and $\zeta$ is the position
of the source. The positions are complex so that we can simplify the 2-dimensional
lens equation into a single complex equation, allowing us to solve it analytically.

Equation (1) can be turned into a complex polynomial of $z$ by
taking its conjugate, substituting the expression for $\bar{z}$ back into
equation (1), moving zeta to the right hand side, and setting the remaining
expression equal to $0$. This results in a 5th-degree  polynomial for the
binary lens case, and a 10th-degree polynomial for the triple lens case.
Writing the equation in terms of a polynomial is favored because it more
easily allows us to use polynomial root solvers to determine the positions
and magnifications of the images. This form of the polynomial, without
further simplifications or assumptions, will hereafter be referred to as
the "general form" of the lens polynomial. The polynomial can alse be derived
and simplified in different coordinate frames, which will be discussed later.

One problem with calculating microlensing events with a typical 64-bit computer
is that floating point values are used to represent these variables. Because
floating point values are expressed in binary code, there will inevitably be
roundoff error as the terms of the polynomial are of greatly varying orders
of magnitude. For reference, a 64-bit double floating point number carries
approximately 16 significant digits. This means, if a computer were to make
the computation $1 + 10^{-17} - 1$, it would truncate the decimal and yield
the value $0$, where we would expect it to return $10^{-17}$. On the other
hand, using more precise calculation methods is not favorable because it will
take much longer than necessary. One purpose of this project is to determine
if we can derive the polynomial form of the lens equation in ways that minimize
the roundoff error without relying on more time-consuming methods. My project
addresses the effects of deriving the polynomial in different coordinate frames
and solving it with different root solver algorithms.

\end{document}
