\documentclass{article}

\usepackage{graphicx}

\begin{document}

Microlensing is becoming an increasingly popular method to search for
planets because of its ability to discover more diverse distributions
of planets than other common methods. Whereas the radial velocity method
and the transit method are most sensitive to hot Jupiters, microlensing
is most sensitive to distance Neptunes. Moreover, it has the potential
to detect Earth-sized and Mercury-sized planets, something the other
methods are not yet able to do. As WFIRST is expected to be launched
in the mid-2020s, one of the goals of the mission is to discover more
properties of planet formation, including for Earth-like planets.

A microlensing event occurs when a distant body, known as the source,
passes behind another body or system of bodies, known as the lens. By
the gravitational lens effect, The light of the source is bent around
the lens and magnified. This produces an observable light curve that can
indicate details about the lensing bodies, including the mass ratios and
separations between each of the lensing bodies. Hereafter, this paper will
focus on point-source point-lens microlensing events for binary and triple
lenses.

In a binary lens, the mass ratio of the planet to the star is denoted by $q$,
and the separation between the planet and the star is denoted by $s$. The
separation is in units of the Einstein radius of the whole lens system, 
which means $q$ and $s$ are both unitless. In a triple lens, there will be two
mass ratios, two separations, and an angle that is formed by the 3 bodies, 
which all have various definitions depending on the types of bodies in the
system (e.g. stars, planets, or a moon). The definitions of these parameters
are arbitrary on a case-by-case basis, so they will not be defined here.

The lensing equation is given by
\begin{equation}
\zeta = z + \sum_{i} \frac{m_{i}}{\bar{z_{i}} - \bar{z}}
\end{equation}
where ${m_{i}}$ is the mass of the lensing body divided by the total mass
of the system, ${z_{i}}$ is the position of the lensing body in units of the
total system's Einstein radius, $z$ is the position of each of the resulting 
images, $\bar{z}$ is the complex conjugate of $z$, and $\zeta$ is the position
of the source. Each of the positions are given in complex space so that the
subsequent calculations can be expressed in one polynomial, rather than 2
separate polynomials in the x- and y-directions. By solving for $z$, taking
the complex conjugate, substituting this expression back into the lensing
equation, and some algebra, we can obtain a very messy polynomial equation
of variable $z$, which is equal to $0$. This results in a 5th-degree
polynomial for the binary lens, and a 10th-degree polynomial for the triple
lens. Writing the equation in terms of a polynomial is favored because it
more easily allows us to use polynomial root solvers to determine the
positions and magnifications of the images. The lensing polynomial can 
alse be derived in different coordinate frames, which will be discussed later.

The problem with calculating microlensing events with a typical 64-bit computer
is that floating point values are used to represent these variables. Because
floating point values are written in binary code, there will inevitably be
roundoff error as the terms of the polynomial are of extremely varying orders
of magnitude. For reference, a 64-bit double floating point number carries
approximately 16 significant digits. This means, if a computer were to make
the computation $1 + 10^{-17} - 1$, it would truncate the decimal and yield
the value $0$, where we would expect it to return $10^{-17}$. On the other
hand, using more precise calculation methods is not favorable because it will
take much longer than necessary. One purpose of this project is to
determine which coordinate frames minimize the effect of the roundoff error,
without relying on more time-consuming methods.

\end{document}
