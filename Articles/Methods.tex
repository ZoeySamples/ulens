\documentclass{article}

\usepackage{graphicx}

\begin{document}

To derive the polynomial equation in a specific coordinate frame, we
must make the appropriate substitutions for the positions of each body.
For example, if the polynomial is derived in the planet frame, we
would define the origin to be the position of the planet, and substitute the
values for each body's position, relative to the origin, into the general form
of the polynomial. Once this substitution is made, the subsequent polynomial is
simplified. This form of the polynomial will hereafter be referred to as the
"specifically-derived" polynomial for some coordinate frame. The coordinate
frames I consider are: the geometric center, planet, star, planetary caustic,
and center-of-mass frames for the binary lens; and the geometric center,
second body (second most massive body), and third body (least massive body) frames
for the triple lens. In each case, the origin is defined as the central position
of the namesake body or location, and all other positions are calcuated relative
to this position. All derivations have been computed via Wolfram Mathematica 7.

[Continued in OriginsInfo.tex]

\end{document}
