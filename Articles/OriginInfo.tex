\documentclass{article}

\usepackage{graphicx}

\begin{document}

Since the purpose of this project is to find which algorithms and
methods result in the most reliable calculations for smaller and
smaller mass ratios, one factor to consider is the way the binary
lens polynomial is derived. The binary lens equation is given by
\begin{equation}
\zeta = z + \frac{m_{1}}{\bar{z_{1}} - \bar{z}} +
\frac{m_{2}}{\bar{z_{2}} - \bar{z}}
\end{equation}
and is the basis of every calculation. This equation can be
manipulated into a fifth-degree polynomial. Depending on
which coordinate frame the calculations will be calculated in,
this equation can be simplified with the appropriate assumptions;
it can also be left in its general form.

In all cases, the two bodies are assumed to lie on the real axis.
The coordinate frames that are being considered are: the geometric
center frame, which assumes the vertical axis lies halfway between
the star and the planet; the planet frame, which assumes the planet
lies on the origin; and the planetary caustic frame, which assumes
the central point of the planetary caustic lies on the origin.

The first test determines which coordinate frames produce the most
reliable results, using both the general polynomial and a version
of the polynomial derived for each coordinate frame. The derivations
were computed by Wolfram Mathematica 7. To test the success of each
calculation, I simulated a grid of points, centered on the planetary 
caustic, to represent the position of the source body. At each point,
I calculated the number of images that should be seen by an observer.
By standard knowledge, there should be three images when the source's
coordinates are outside the caustic, and five images when the source's
coordinates are inside the caustic. \textbf{Figure 1} and 
\textbf{Figure 2} show the results of this simulation.

\begin{figure}
	\includegraphics[width=0.9\textwidth]{../Tables/test_SFD_0.png}
	\caption{Place caption text here.}

	\vspace*{\floatsep}

	\includegraphics[width=0.9\textwidth]{../Tables/test_SFD_1.png}
	\caption{Place caption text here.}
\end{figure}



When using the general form of the polynomial equation, the planet frame
performs better than the geometric center frame and the caustic frame only
when the mass ratio approaches $10^{-8}$; at mass ratios greater than $10^{-7}$,
none of the coordinate frames produce any errors. However, all of the frames
result in high numbers of errors when the mass ratio drops below something
of the order $10^{-8}$.

On the other hand, when using the specifically-derived forms of the
polynomial, we see much different results. The planet frame now produces
a correct plot of the number of images all the way down to a mass ratio of
$10^{-15}$; whereas the other frames' performance is slightly worse. The
hypothesized reason for this is that the form of the polynomial specifically
derived for the planet frame elimates many instances of terms with the
mass ratio squared. Because 64-bit floating point numbers can only store
around 16 digits, many errors tend to accumulate when adding terms of the
order $10^{0}$ with other terms of the order $10^{-16}$. By eliminating
nearly every instance of the mass ratio squared, it is possible to keep
every significant digit throughout the calculation with mass ratios all
the way down to $10^{-15}$.

\end{document}
