\documentclass{article}

\usepackage{graphicx}

\begin{document}

% The following part of this document will be placed in the Methods section

The first test determines, for the binary lens, which coordinate systems
and which version of the derivation (general versus specific) ought to be
used to produce reliable results as the mass ratio decreases. In all cases,
the two bodies are assumed to lie on the real axis. The coordinate frames
that are being considered are: the geometric center frame, which assumes
the vertical axis lies halfway between the star and the planet; the planet
frame, which assumes the planet lies on the origin; and the planetary
caustic frame, which assumes the central point of the planetary caustic
lies on the origin. 

To test the success of each calculation, I simulated a grid of points
centered on the planetary caustic. Each point represents a position
for the source body. At each point, I calculated the number of images
that should be seen by an observer. There should be three images when
the source is outside the caustic, and five images when the source is
inside the caustic. We can make these plots and check whether they
produce a result that agrees with what we expect. However, checking whether
the frames yield the correct number of solutions is insufficient to qualify
whether they pass. I also calculate the magnification at each of the
points in the grid. These simulations are done for each  of the
aforementioned coordinate frames, for both the specifically-derived and
general forms of the polynomial. Success is determined by whether the
plots produce results that agree with theoretical expectations. This can
be qualified visually, as it is clear when a method fails because the
plot will reproduce a very noisey result with seemingly no regard to the
caustic. \textbf{Figure 1} and \textbf{Figure 2} show the plots of the
number of accepted images; \textbf{Figure 3} and \textbf{Figure 4} show
the plots of the magnification for the same grid of points.

\begin{figure}
	\includegraphics[width=0.9\textwidth]{../Tables/images_SFD_0.png}
	\caption{The number of images versus position for each coordinate
	frame using the general form of the polynomial. The top row shows
	the plots for a mass ratio, $q=10^{-6}$. The bottom row shows the
	plots for a mass ratio, $q=10^{-12}$. Notice that the scale on the
	x- and y-axes differ by orders of magnitude. There are two values of
	separation for the planet frame; the reason will be justified in
	the next plot. As expected, for $q=10^{-6}$, the plots show the
	correct number of images for each coordinate system. However, for
	$q=10^{-12}$, the plots fail for all coordinate frames.}

	\vspace*{\floatsep}

	\includegraphics[width=0.9\textwidth]{../Tables/images_SFD_1.png}
	\caption{Same as \textbf{Figure 1} except using the specifically-
	derived forms of the polynomial. As
	expected, for $q=10^{-6}$, the plots show the correct number of
	images for each coordinate system. For $q=10^{-12}$, the plots fail
	for all coordinate frames except for the planet frame. Because it
	succeeds for planet frame with a separation $s>1$, it it also worth
	checking if it succeeds for a separation, $s<1$. It does succeed,
	and justifies that the planet frame is the only frame that succeeds
	for very small mass ratios.}
\end{figure}

\begin{figure}
	\includegraphics[width=0.9\textwidth]{../Tables/magn_SFD_0.png}
	\caption{The magnification versus the position for each coordinate
	frame using the general form of the polynomial. As seen in plots
	of the number of images, the simulations pass for all coordinate
	frames when $q=10^{-6}$, and fail for all coordinate frames when
	$q=10^{-12}$.}

	\vspace*{\floatsep}

	\includegraphics[width=0.9\textwidth]{../Tables/magn_SFD_1.png}
	\caption{Same as \textbf{Figure 1} except using the specifically-
	derived forms of the polynomial. As
	seen in the plots of the number of images, when $q=10^{-12}$, the
	simulation fails for all coordinate frames except for the planet
	frame. Again, it succeeds for both $s>1$ and $s<1$ frame.}
\end{figure}

% The following part of this document will be placed in the Results/Analysis
% section

When using the general form of the polynomial equation, the planet frame
performs better than the geometric center frame and the caustic frame only
when the mass ratio approaches $10^{-8}$; at mass ratios greater than
$10^{-7}$, none of the coordinate frames produce any errors. However, all
of the frames result in high numbers of errors when the mass ratio drops
below something of the order $10^{-8}$.

On the other hand, when using the specifically-derived forms of the
polynomial, we see much different results. The planet frame now produces
a correct plot of the number of images all the way down to a mass ratio of
$10^{-15}$; whereas the other frames' performance is slightly worse. The
hypothesized reason for this is that the form of the polynomial specifically
derived for the planet frame elimates many instances of terms with the
mass ratio squared. Because 64-bit floating point numbers can only store
around 16 digits, many errors tend to accumulate when adding terms of the
order $10^{0}$ with other terms of the order $10^{-16}$. By eliminating
nearly every instance of the mass ratio squared, it is possible to keep
every significant digit throughout the calculation with mass ratios all
the way down to $10^{-15}$.

\end{document}
